\chapter{CLOSING}
\thispagestyle{fancy}

%====================================================================================
	\section{Conclusions}
	    \iffalse
	    First, we discuss how to derive the governing equations of fluid dynamics related to blood flow. In section 2, we start from an introduction to Reynold's transport theorem, and we demonstrate how it relates to the derivations of the conservation law equations. Furthermore, the influence of the assumptions on the equations is also discussed.
	    \par
	    Second, we discuss how to derive the SPH numerical schemes for the governing equations. In section 3, we first discuss the underlying idea of SPH approximations. Then, we discuss about the properties of the smoothing functions and the choices of the smoothing functions. After that, we derive the SPH numerical schemes and discuss about time integration.
	    \par
	    Last, we discuss how to develop the code and the graphical simulation. In section 4, we first discuss about the boundary models. After that, we discuss how to achieve a steady flow in the numerical simulation. Then, we present the design of the graphical simulation. In section 5, we discuss about the results and implementation. We also evaluate the performance of the code, and we describe ways to optimize performance.
	    
        The governing equations of fluid dynamics related to blood flow in the artery has been successfully derived. The Reynold's transport theorem is used to derive the conservation laws equations. The result is the incompressible Navier-Stokes equation with constant viscosity. \par
	    We use the SPH method to numerically solve the governing equations. We discussed the underlying idea of SPH approximations. We discussed about the properties of the smoothing functions and the choices of the smoothing functions. We derived equations to easily evaluate the derivatives of the smoothing functions, and we presented proofs that the chosen smoothing functions satisfy the properties. We developed the numerical schemes by directly discretizing the Navier-Stokes equation using the SPH method. \par
	    The code and the graphical simulation are implemented within the MATLAB R2013a programming environment. The user interface is created using the MATLAB's GUIDE. The performance analysis results showed that preallocation significantly affects the run time. The boundary treatment method together with the inlet and outlet boundary conditions are used to achieve a steady flow. The numerical simulation results showed that a steady flow has been successfully achieved. The flow in the pulsating artery case is pulsatile. In the pulsating stenosed artery case, the pressures of the particles at the narrowed part are higher than the nonpulsating stenosed artery case. 
        \fi
        
        A new deep reinforcement learning model for video games has been successfully developed. Deepmind's Deep Q-Learning and Adam optimizer are combined to allow deep reinforcement learning to be used in a more complex video games. The result is Adam-DQL, a deep reinforcement learning model that handles immense state spaces extremely well. To further improves stability, a new training/improvement policy on deep-reinforcement learning is required. Partial training, a technique based on the idea of kickstarting an agent to experience more rewards/punishment is developed. A similar technique proposed for robotic environment, called demonstrations, is also applied on Adam-DQL. 
        \par
        Adam-DQL agent is created within Tensorflow environment on top of Python 2.7.13. Several framework architectures are considered and tested. Testing environment based on video games are proposed and used to test Adam-DQL's capability to solve different type of video games. Adam-DQL agents are tested on 3 games, with different type and complexity. The results is then plotted and compared to classical deep Q-Learning. Adam-DQL shows superior results from both training speed and reward standpoint. On some games, Adam-DQL even shows really drastic improvements. Demonstration helps Adam DQL agent to start improving pretty fast, and combined with partial training, Adam-DQL agent can solve harder games that's rarely touched by deep reinforcement learning before.   
%====================================================================================
	\section{Future Works}
		\iffalse
		The computational cost of the code is still expensive. The SPH method's computational complexity significantly depends on the algorithm used for nearest neighbor search. In this thesis, the exhaustive search is used. The exhaustive search is very easy to implement, but it is the most computationally expensive nearest neighbor search algorithm. The computational complexity of the exhaustive search is $O(N^2)$, where $N$ is the number of particles. Thus, applying a better nearest neighbor search algorithm such as neighbour lists is strongly recommended. \par
		To simulate the behavior of solid boundaries, the use of appropriate boundary treatment methods is essential. Using the boundary treatment method described in section 4.1.2, we found that we need a smaller time step for the pulsating artery case to prevent penetration. A smaller time step implies more expensive computational cost. Hence, an adaptive time step approach is suggested. Another approach is to mathematically solve for the intersection point between the straight trajectory of the particle and the boundary equation. Then, the direction of the particle is directly inverted to simulate the collision. \par
	    In this thesis, blood is assumed as a Newtonian fluid. As described in section 2.3, the Newtonian fluid assumption is valid for blood in larger arteries such as the coronary arteries. Hence, it is suggested that the model is further developed for blood in smaller arteries where the non-Newtonian behavior is significant. Furthermore, the model can also be further developed by taking into account the elasticity of the blood vessel. \par
		The developed model can also be used for wider applications. It is suggested that the model is used for estimating the risk of artery wall rupture. The graphical user interface can also be improved. When a pop-up message is displayed or when the progress bar is displayed, the user interface should be disabled. Furthermore, the parameter to determine the degree of stenosis has not been implemented in the user interface yet. It is suggested that the relationship between the parameter to determine degree of stenosis and the time step required to prevent penetration is also determined.
		\fi
		Adam-DQL agent in this thesis are specifically developed for pyGame environment. To allow applications in wider variety of games, it is suggested that the agent are created with any video games environment in mind. The use of faster language (C++) is also recommended as performance is critical for video games. The software framework in general can also be improved, by converting the general structure from a reusable class to a shared libraries, that will allow easier testing and distribution for further research. Furthermore, a stronger hardware for training is strongly recommended.
		\par
		In this thesis, a simplified Tetris environment, consisting of only 2 blocks is used. Based on the idea of training the agent on the simplest subproblems first, further training by using partial training for the rest of the Tetris blocks in this specific order: $\{L,T,S\}$, a fully performing Tetris agent can be developed. There's also 2 unsolved problem for this environment, which is blocking the further improvement of the Tetris environment, which hopefully can be solved in the future.
		\par
		The developed Adam-DQL model can also be used for wider applications. Despite being designed specifically for video games, the model itself can be used for almost every reinforcement learning tasks. For example, by feeding an image of stock prices movements to the neural network, an automated stock training agent can be developed. Furthermore, Adam-DQL agents can also be used in robotic tasks as it combines movement (action) and vision (states). 
		